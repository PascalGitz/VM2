\chapter*{Kurzfassung}

Diese Arbeit ist Teil einer Arbeitsserie, bestehend aus dem Vertiefungsmodul I, dem Vertiefungsmodul II und der Masterthesis. Der Kern dieser Serie ist die Beschreibung von Verformungen im Stahlbetonbau basierend auf mechanischen Modellen. In der ersten Vertiefungsarbeit wurden die Hintergründe der Modelle beleuchtet und mittels Handrechnungen an Versuchen verifiziert. In dieser Arbeit, dem Vertiefungsmodul II, werden die Ansätze aus der Vorarbeit mit kommerziellen FEM-Tools kombiniert. Es werden die Softwares AxisVM-X7, RFEM-6 und IDEAStatiCa verwendet.

Ein Modellvorstellung, das Federmodell, wird aufgezeigt. Dabei werden biege- und schubstarre Stäbe in regelmässigen Abständen durch Federn gekoppelt, in denen beliebig nichtlineare Steifigkeitsbeziehungen hinterlegt werden können. Dies ermöglicht die Bestimmung nichtlinearer Verformungen. Die Steifigkeiten werden anhand der Momenten-Krümmungs-Beziehung und anhand der Steifigkeit der Schubbewehrung ermittelt. Das Modell wird an einem Einführungsbeispiel illustriert, wobei analytische Lösungen mit den FEM-Ergebnissen verglichen werden. Diese zeigen, dass die Ergebnisse nahezu deckungsgleich und zufriedenstellend sind, mit minimalen numerischen Abweichungen, die auf numerische Approximationen zurückzuführen sind.

Im Anschluss wird das Modell verifiziert. In diesem Kapitel werden die Versuche aus der Vorarbeit mit dem Federmodell nachgerechnet, um identische Ergebnisse wie bei den Handrechnungen zu erzielen. An einem Dreipunktbiegeversuch werden Schub- und Biegeverformungen basierend auf nichtlinearen Stoffgesetzen ermittelt und mit den gemessenen Versuchsresultaten verglichen. Die Ergebnisse sind zufriedenstellend. An einem Vierpunktbiegeversuch werden ebenfalls Biege- und Schubverformungen bestimmt und die Anwendung des Versatzmasses aufgezeigt. Auch hier werden die Modellresultate mit den Versuchsresultaten verglichen, wobei Abweichungen erkennbar sind, die knapp akzeptabel sind. Diese Unterschiede werden auf Unschärfen in den angewendeten Stoffgesetzen zurückgeführt. Im Vergleich mit den Handrechnungen aus der Vorarbeit zeigt das Federmodell jedoch identische Ergebnisse. Die Modellverifizierung wird abschliessend als erfolgreich betrachtet.

Im Kapitel der Modellanwendung wird das Federmodell an einem vorgespannten Träger angewendet. Das Kapitel beginnt mit einer detaillierten Versuchsbeschreibung, gefolgt von der Modellbildung. Die Interpretation der Vorspannung als Einwirkung sowie der Lastfall der Pressenkräfte und das Eigengewicht werden beschrieben. Der Hauptaspekt der Analyse ist die Ermittlung der Momenten-Krümmungs-Beziehung anhand nichtlinearer Stoffgesetze. Mittels einer Querschnittssoftware werden entlang der Stabachse 136 unterschiedliche Querschnitte analysiert. Die numerischen Ergebnisse werden erfolgreich durch Handrechnungen verifiziert. Die Resultate des Modells werden mit den Versuchsmessungen verglichen, wobei ein annähernd deckungsgleicher Verlauf festgestellt wird. Lediglich im Bereich der Traglast sind Unterschiede erkennbar. Das Kapitel endet mit einem Vergleich des Federmodells mit der Compatible Stress Field Method (CSFM), die in IDEAStatiCa implementiert ist. Die CSFM zeigt ähnliche Resultate.
